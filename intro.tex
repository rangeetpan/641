% $Id: intro.tex,v 1.1.1.1 2007/10/03 22:28:18 hridesh Exp $

\section{Introduction}

The abstract was a verbatim excerpt from Phil Koopman's advice 
on writing abstract~\cite{koopman97a}. The copyright for the 
text is held by Philip Koopman. More information about it is 
available in the reference. Below we will discuss Dr. Stirewalts's
5-paragraph rule for writing introductions. 

{\bf Dr. Stirewalt's 5-paragraph rule for writing Introductions}: 
Of the many tasks involved in writing a good conference
paper, I find writing the introduction section to be
the most difficult. This is unfortunate, as a poorly
structured argument sets the wrong tone for what might
otherwise be really good research. To help manage this
painful process, I have developed a heuristic, called the
{\em five-paragraph rule}, that is useful for organizing
introductions. The heuristic prescribes that good
introductions should contain a sequence of five major
pieces, each of which should fit into a single paragraph
in order to force the writer to communicate at the
appropriate level of abstraction. The heuristic borrows
ideas from {\em persuasive argument} and
{\em structured analysis/structured design} (ala DeMarco/Yourdon),
and it is reminscent of a similar structuring mechanism
from freshman level courses in English composition.
My success in publishing papers increased dramatically
once I began to use this heuristic to structure my
introductions.

The heuristic is: Design your introductions to comprise five paragraphs
whose purpose and contents are as follows:

\begin{enumerate}
\item {\bf Introductory paragraph}: Very briefly: What is the problem and
        why is it relevant to the audience attending *THIS CONFERENCE*?
        Moreover, why is the problem hard, and what is your solution?
        You must be brief here. This forces you to boil down your
        contribution to its bare essence and communicate it directly.

\item {\bf Background paragraph}: Elaborate on why the problem is hard,
        critically examining prior work, trying to tease out one
        or two central shortcomings that your solution overcomes.

\item {\bf Transition paragraph}: What keen insight did you apply
        to overcome the shortcomings of other approaches?
        Structure this paragraph like a syllogism:
        Whereas P and P => Q, infer Q.

\item {\bf Details paragraph}: What technical challenges did you have
        to overcome and what kinds of validation did you perform?

\item {\bf Assessment paragraph}: Assess your results and briefly state
        the broadly interesting conclusions that these results
        support. This may only take a couple of sentences. I
        usually then follow these sentences by an optional
        overview of the structure of the paper with interleaved
        section callouts.
\end{enumerate}

