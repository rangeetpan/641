\section{Conclusion and Future Works}
\label{sec:future}
In this study, we have found the random initializations for which accuracy i.e., trustworthiness changes even if the experimental setup remains unchanged. After identifying these randomly initialized parameters, we have applied a search algorithm using user-provided maximum allowable time, trial, and gain from the actual accuracy in order to achieve the near-optimal accuracy. Our prior choice of the search process was the Simulated Annealing process. However, from the empirical results depicted in Figure \ref{fig:delta}, we have found that the search space of the accuracy based on the weight and bias initialization is not smooth. Because of that, the algorithm fells into the local maxima for several times and does not produce a good result. Other challenges while implementing this approach is that we do not find any pattern of the trial and time with the gain that can help us to direct the searching process. Though this could be a research question to answer in the future, we have addressed the choice of the algorithm and the stopping point of the process based on the user intent.
Thus, a user intent based near-optimal accuracy searching strategy has been introduced to restrict the learning process that makes a DNN model accountable in terms of classification accuracy. Though currently, our work reruns the training process by modifying the weight and bias, the resource cost for the work is high and while, it has been proved that current initialization of the weight and bias has the potentiality to increase the accuracy, however, how efficient this process over random initialization is an open question that could be an interesting avenue to venture. Also, in the future, we can extend by integrating our approach with AutoKeras \cite{jin2019auto} and validate how this proposed approach can increase the accuracy over the output of the AutoKeras.

